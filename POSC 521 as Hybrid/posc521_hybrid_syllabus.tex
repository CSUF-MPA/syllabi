\documentclass[12pt, letterpaper]{article}
\usepackage[english]{babel}
\usepackage[T1]{fontenc}
%\usepackage[light]{ubuntu}
\usepackage[margin=1.25in]{geometry}
\usepackage{xcolor}
\usepackage{url}
\usepackage[utf8]{inputenc}
\usepackage{tabularx}
\usepackage{booktabs}
\frenchspacing
\usepackage{multicol}
\usepackage{eso-pic}
\usepackage[longnamesfirst]{natbib}
\bibpunct{(}{)}{;}{a}{}{,}
\usepackage{caption}
\usepackage{subcaption}
\usepackage{setspace}
\usepackage{paralist}
\usepackage{quoting}
\usepackage{comment}
\usepackage{enumitem}
\usepackage{graphicx}
\usepackage{float}
\usepackage{bookmark}
\renewcommand{\thesection}{\arabic{section}.}
\renewcommand{\thesubsection}{\thesection\arabic{subsection}}
\renewcommand{\thesubsubsection}{\thesubsection.\arabic{subsubsection}}
\usepackage{hyperref}
\hypersetup{
    colorlinks=true,
    linkcolor=blue,
    filecolor=magenta,      
    urlcolor=cyan,
}

\begin{document}
\title{MPA Capstone Seminar: \\ Public Administration Theory}


\author{POSC 588 — Fall 2024}
\date{\textbf{Hybrid Course} \\ 
\vspace*{0.5em}
In-Person Sessions: \\ 
\vspace{0.5em}
Synchronous Online Sessions: \\
\vspace{0.5em}
Asynchronous Online Sessions:
}

    \maketitle


    \subsection*{Professor: David P. Adams, Ph.D.}

    \subsubsection*{Contact Information:}
    
    \begin{itemize}
        \item Office: 516 Gordon Hall
        \item Phone/SMS: (657) 278-4770
        \item website: \href{https://dadams.site}{\texttt{https://dadams.site}}
        \item email: \href{dpadams@fullerton.edu}{\texttt{dpadams@fullerton.edu}}
        \item Office hours: Tuesdays \& Thursdays from 9:30 to 11:00, Thursdays from 5:30 to 6:30, and by \href{https://t.ly/dpa-appt}{appointment}.
        \item Schedule meetings throughout the week: \href{https://t.ly/dpa-appt}{\texttt{https://t.ly/dpa-appt}}
    \end{itemize}
    
    \section{Catalog Description}
    Concepts, models and ideologies of public administration within the larger political system. Course restricted to students in their final six units of graduate work.
    
    \section{Course Description}
    The capstone seminar in the Master of Public Administration program at Cal State Fullerton examines concepts, models, and ideologies of public administration within the larger political system.
    
    \section{Course Objectives}
    This course is designed to accomplish five interrelated objectives:
    
    \begin{enumerate}
        \item \textbf{Theory Examination}: We will delve into the most important theories and literature in public administration, fostering a deep understanding of the field.
        \item \textbf{Literature Review}: You will complete a literature review in your concentration area, allowing you to specialize and delve deeper into a specific aspect of public administration. This preparation will be crucial for the general concentration portion of the comprehensive exams.
        \item \textbf{Writing Skills}: This course will enhance your writing skills, focusing on clear, concise, and effective communication. This preparation will be crucial for the general theory portion of the comprehensive exams.
        \item \textbf{Peer Review and Collaboration}: Through a structured peer review process, you will learn to provide constructive feedback, gain new perspectives, and improve your own work based on your peers’ insights. This collaborative learning approach is designed to mimic real-world public administration environments where collaboration and feedback are key.
    \end{enumerate}
    
    \subsection*{Course Topics}
    Throughout the course, we will cover the following topics:
    \begin{table}[h]
        \centering
        \caption{Public Administration Topics}
        \begin{tabular}{ll}\hline
            Public Administration Theory & Public Administration in the U.S. Context \\
            Street-Level Bureaucrats & Public Policy and Implementation \\
            Public Service Values and Ethics & Privatization and Contracting \\
            Leadership and Management & Performance Management \\
            Motivation & Nonprofits and Public Administration \\
            The Future of Public Administration & Current Issues in Public Administration \\ \hline
        \end{tabular}
        \label{tab:pa_topics}
        \end{table}
    
    \section{Course Materials}
    \subsection*{Required Texts}
    \begin{itemize}
        \item \textbf{Denhardt and Denhardt}. \textit{The New Public Service: Serving, Not Steering}. 4th ed. Routledge, 2015.
        \item \textbf{Gooden, Susan T.}. \textit{Race and Social Equity: A Nervous Area of Government}. Oxford University Press, 2014.
        \item \textbf{Kamarck, Elaine C.}. \textit{The End of Government…As We Know It: Making Public Policy Work}. 2nd ed. Lynne Rienner Publishers, 2015.
        \item \textbf{Lipsky, Michael}. \textit{Street-Level Bureaucracy: Dilemmas of the Individual in Public Services}. Russell Sage Foundation, 2010.
    \end{itemize}
    
    \begin{itemize}
        \item Denhardt and Denhardt (2015) develop a framework emphasizing the importance of public service values, democratic engagement, and collaboration between citizens and government. The authors argue that the New Public Service (NPS) is a departure from the traditional public administration model, which prioritizes efficiency and effectiveness and instead emphasizes the public interest, social equity, and the need for public servants to engage with citizens and communities. The NPS emphasizes that public service is a calling and requires a commitment to serving the public good above personal gain or profit. The framework has implications for how public organizations are structured, how they engage with citizens, and how they are held accountable for their actions.
        \item Gooden (2004) explores the historical and contemporary challenges surrounding race and social equity within public administration. The book highlights the importance of addressing racial disparities in policy outcomes and emphasizes the critical role of public administrators in promoting social equity. Through a comprehensive analysis of the intersection between race and public policy, Gooden offers practical strategies and best practices for fostering a diverse and inclusive public service workforce. This essential resource equips students and practitioners with the knowledge and tools necessary to confront racial disparities and work towards a more equitable society in their roles as public administrators.
        \item Kamarck (2007) posits that the traditional bureaucratic model of government is evolving into a more efficient and responsive system due to the transformative power of technology and globalization. She contends that this shift is characterized by the rise of networked governance, in which governments increasingly collaborate with private and non-governmental organizations to deliver public services and adopt performance-based management techniques. While challenging the status quo, Kamarck argues that these changes enhance governments’ capacity to address complex social and economic issues, providing citizens with more effective and accountable institutions.
        \item Lipsky (2010) focuses on the role of front-line public service workers, or ``street-level bureaucrats,'' who directly interact with the public and implement policies. He explores the dilemmas these workers face, such as limited resources, conflicting goals, and the need to exercise discretion in decision-making. The book relates to public administration by illuminating the importance of understanding the experiences and challenges faced by street-level bureaucrats as their actions determine the success or failure of public policies.
    \end{itemize}


\subsection*{Additional Readings are indicated in the course schedule below.}


\section{Technical Problems}

\subsection*{University IT Help Desk}

Contact the instructor immediately to document the problem if you encounter any technical difficulties. Then contact the \href{http://www.fullerton.edu/it/students/helpdesk/index.php}{Student IT Help Desk} for assistance. You can also call the Student IT Help Desk at (657) 278-8888, \href{mailto:StudentITHelpDesk@fullerton.edu}{email}, visit them at the Pollak Library North \href{http://www.fullerton.edu/it/students/sgc/index.php}{Student Genius Center}, or log on to the \href{http://my.fullerton.edu/}{my.fullerton.edu} portal and click ``Online IT Help'' followed by ``Live Chat''.

\subsection*{Canvas Support}

If you encounter any technical difficulties with Canvas, call the Canvas Support Hotline at 855-302-7528, visit the \href{https://community.canvaslms.com/docs/DOC-10720-67952720329}{Canvas Community}, or click the ``Help'' button in the lower left corner of Canvas and select ``Report a Problem''. The \href{https://cases.canvaslms.com/liveagentchat?chattype=student&sfid=001A000000YzcwQIAR}{Student Support Live Chat} is available 24 hours a day, 7 days a week.


\section{University Student Policies}

In accordance with UPS 300.00, students must be familiar with certain policies applicable to all courses. Please review these policies as needed and visit this Cal State Fullerton website \href{https://t.ly/csuf-syllabus}{https://t.ly/csuf-syllabus} for links to the following information:

\begin{enumerate}
    \item   University learning goals and program learning outcomes.
    \item	Learning objectives for each General Education (GE) category.
    \item	Guidelines for appropriate online behavior (netiquette).
    \item	Students’ rights to accommodations for documented special needs.
    \item   Campus student support measures, including Counseling \& Psychological Services, Title IV and Gender Equity, Diversity Initiatives and Resource Centers, and Basic Needs Services.
    \item	Academic integrity (refer to UPS 300.021).
    \item	Actions to take during an emergency.
    \item	Library services information.
    \item	Student Information Technology Services, including details on technical competencies and resources required for all students.
    \item	Software privacy and accessibility statements.
\end{enumerate}

\section{Course Student Policies}

\subsection*{Course Communication}
All course announcements and communications will be sent via \emph{Canvas} and university email. Students are responsible for regularly checking their \emph{Canvas} notifications and email. Students are also responsible for ensuring that their \emph{Canvas} notifications are set to receive messages from the course. Students are expected to check \emph{Canvas} and their email at least once daily.

\textbf{Response Time}: I will strive to respond to all student emails and \emph{Canvas} messages within 24 hours, except on weekends and holidays. If you do not receive a response within 24 hours, please send a follow-up message. If you do not receive a response within 48 hours, please send another follow-up message and contact me via phone or SMS text at (657) 278-4770.

\subsection*{Due Dates}
Please know that exams are only permitted on the scheduled date as indicated in the course schedule below. If you have concerns about meeting assignment deadlines, please contact the professor in advance to discuss potential accommodation.

\subsection*{Alternative Procedures for Submitting Work}
Students are expected to submit all assignments via \emph{Canvas}. If you cannot submit an assignment via \emph{Canvas}, please contact the professor to discuss alternative submission procedures.

\subsection*{Extra Credit}
An extra credit opportunity exists for each of the weekly synthesis assignments. Students can receive up to an additional five points for completing a reflection on the assignment using the criteria indicated below. Please do not ask for additional extra credit assignments.

\subsection*{Academic Integrity}
Students are expected to adhere to the highest standards of academic integrity. Any student found to have engaged in academic dishonesty will be subject to the sanctions described in the \href{https://www.fullerton.edu/senate/publications_policies_resolutions/ups/UPS%20300/UPS%20300.021.pdf}{Academic Dishonesty Policy} (UPS 300.021). Academic dishonesty includes, but is not limited to, cheating, plagiarism, fabrication, facilitating academic dishonesty, and submitting previously graded work without prior authorization. Students are expected to be familiar with the university's policy on academic dishonesty and to adhere to this policy in all aspects of this course. Any student who has questions about the policy should ask the professor for clarification.


\section{Course Requirements}


\section{Grades}

Your work in this class will be graded based on four criteria:
    \begin{enumerate}
        \item Thoroughly complete each assignment, address all questions, and participate in class discussions.
        \item Effective use of class materials (and other literature while researching your literature review topic).
        \item Sophisticated substantive content and discussion rather than superficial.
        \item Writing at the graduate level, including proper mechanics, grammar, syntax, and citation style.
    \end{enumerate}

\subsection*{Grading Scale and Grade Weights}  

The grading scale is shown in Table~\ref{tab:grading-scale}. Grades will be given based on the weights in Table~\ref{tab:grade-weights}.

\begin{table}[h]
\centering
\caption{Grading Scale}
\begin{tabular}{llll}
\toprule
\textbf{Grade} & \textbf{Percentage} & \textbf{Grade} & \textbf{Percentage} \\
\midrule
A+ & 98.0 -- 100 & B- & 80.0 -- 81.9\\
A & 92.0 -- 97.9 & C+ & 78.0 -- 79.9\\
A- & 90.0 -- 91.9 & C & 72.0 -- 77.9\\
B+ & 88.0 -- 89.9 & C- & 70.0 -- 71.9\\
B & 82.0 -- 87.9 & & \\
\bottomrule
\end{tabular}
\label{tab:grading-scale}
\end{table}


\begin{table}[h!]
\centering
\caption{Grade Weights}
\begin{tabular}{ll}
    \toprule
\textbf{Assignment} & \textbf{Percentage} \\
\midrule
Item 1 & 60\% \\
Item 2 & 25\% \\
Item 3 & 10\% \\
Item 4 & 5\% \\
\bottomrule
\end{tabular}
\label{tab:grade-weights}
\end{table}

\section{Course Schedule}

\end{document}