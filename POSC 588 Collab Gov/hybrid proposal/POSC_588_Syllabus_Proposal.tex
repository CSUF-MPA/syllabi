\documentclass[12pt, letterpaper]{article}
\usepackage[english]{babel}
\usepackage[T1]{fontenc}
\usepackage[margin=1in]{geometry}
\usepackage[light]{ubuntu}
\usepackage{xcolor}
\usepackage{url}
\usepackage[utf8]{inputenc}
\usepackage{tabularx}
\usepackage{booktabs}
\frenchspacing
\usepackage{multicol}
\usepackage{eso-pic}
\usepackage[longnamesfirst]{natbib}
\bibpunct{(}{)}{;}{a}{}{,}
\usepackage{caption}
\usepackage{subcaption}
\usepackage{setspace}
\usepackage{paralist}
\usepackage{quoting}
\usepackage{comment}
\usepackage{enumitem}
\usepackage{graphicx}
\usepackage{float}
\usepackage{bookmark}
\renewcommand{\thesection}{\arabic{section}.}
\renewcommand{\thesubsection}{\thesection\arabic{subsection}}
\renewcommand{\thesubsubsection}{\thesubsection.\arabic{subsubsection}}
\usepackage{etoolbox}
\patchcmd{\thebibliography}{\section*{\refname}}{\section{\refname}}{}{}
\usepackage{hyperref}
\hypersetup{
    pdftitle={Syllabus: POSC_588 Fall 2024}, 
    pdfauthor={David P. Adams, Ph.D.},
    pdfsubject={MPA Collaborative Governance},
    pdfkeywords={syllabus, 5888, MPA, Adams, CSUF, POSC, collaboration, theory, collaborative governance, AI, ChatGPT},
    pdfproducer={VS Code 1.60.2}, 
    pdfcreator={pdflatex}}



\begin{document}
\title{Collaborative Governance}

\author{POSC 588 — Fall 2024}
\date{Hybrid Course: \\ \vspace*{0.5em}
    In-Person Sessions on Tuesday Aug. 27 and Oct 29,\\7:00–9:45pm, Room TBD \\\vspace{0.5em}
    Synchonous Online Tuesdays 7:00–9:45pm, \href{https://fullerton.zoom.us/j/1234567890}{https://fullerton.zoom.us/j/1234567890}}

    \maketitle


\subsection*{Professor: David P. Adams, Ph.D.}

\subsubsection*{Contact Information:}

\begin{itemize}
	\item Office: 516 Gordon Hall
	\item Phone/SMS: (657) 278-4770
	\item website: \href{https://dadams.io}{\texttt{https://dadams.io}}
	\item email: \href{dpadams@fullerton.edu}{\texttt{dpadams@fullerton.edu}}
	\item Office hours: Tuesdays \& Thursdays from 9:30 to 11:00, Thursdays from 5:30 to 6:30, and by \href{https://t.ly/dpa-appt}{appointment}.
	\item Schedule meetings throughout the week: \href{https://t.ly/dpa-appt}{\texttt{https://t.ly/dpa-appt}}
\end{itemize}


\section{Technical Problems}

\subsection*{University IT Help Desk}

Contact the instructor immediately to document the problem if you encounter any technical difficulties. Then contact the \href{http://www.fullerton.edu/it/students/helpdesk/index.php}{Student IT Help Desk} for assistance. You can also call the Student IT Help Desk at (657) 278-8888, \href{mailto:StudentITHelpDesk@fullerton.edu}{email}, visit them at the Pollak Library North \href{http://www.fullerton.edu/it/students/sgc/index.php}{Student Genius Center}, or log on to the \href{http://my.fullerton.edu/}{my.fullerton.edu} portal and click ``Online IT Help'' followed by ``Live Chat''.

\subsection*{Canvas Support}

If you encounter any technical difficulties with Canvas, call the Canvas Support Hotline at 855-302-7528, visit the \href{https://community.canvaslms.com/docs/DOC-10720-67952720329}{Canvas Community}, or click the ``Help'' button in the lower left corner of Canvas and select ``Report a Problem''. The \href{https://cases.canvaslms.com/liveagentchat?chattype=student&sfid=001A000000YzcwQIAR}{Student Support Live Chat} is available 24 hours a day, 7 days a week.

\section{Response Time} I will strive to respond to all student emails and \emph{Canvas} messages within 24 hours, except on weekends and holidays. If you are still awaiting a response within 24 hours, please send a follow-up message. If you are still waiting to receive a response within 48 hours, please send another follow-up message and contact me via phone or SMS at (657) 278-4770.

\section{Catalog Description}

Topics include federalism, intersectoral public administration, intergovernmental relations, public-private partnerships, public contract management, interlocal agreements, and network governance.

\section{Course Description}

Organizations across all sectors increasingly respond to complex problems through involvement in networks that offer innovative and flexible responses. Managing networks is different from managing a single organization. Knowing ways of working within and across organizations is essential to effective performance in a networked system. This course focuses on collaborative governance as interactions across nonprofit, for-profit, and public sectors, with analyses and applications. The course also focuses on federalism, intergovernmental relations, public-private partnerships, contract management, interlocal service provision and production, and networked governance.

\section{Course Objectives}

While collaborative governance can help generate and implement enduring and meaningful public policy, it can also be challenging. This course explores the management issues raised by collaborative governance. It seeks to provide a theoretical and practical foundation so that you can become a better producer and consumer of the processes, tools, and approaches to collaborative governance. By the end of the course, students should be able to

\begin{enumerate}
    \item Identify fundamental changes in public management that have led to the increasing usage of intergovernmental, interagency, and intersectoral networks;
    \item Understand the difference between managing hierarchies and managing networks;
    \item Practice and apply various techniques and tools for improving collaborative governance;
    \item Suggest courses of action for improving the performance of collaborative governance;
    \item Describe key concepts, principles, tools, and problems associated with collaborative governance;
    \item Demonstrate how collaborative governance is being used to address contemporary issues and assess the potential of collaborative governance for modern policy problems. 
\end{enumerate}


\section{Required Texts}

There are two textbooks for this course:

\begin{enumerate}[leftmargin=!,labelindent=5pt,itemindent=-15pt]
    \item Agranoff, Robert. 2012. \emph{Collaborating to Manage: A Primer for the Public Sector}. Washington, DC: Georgetown University Press.

    \item  Henderson, Alexander C. 2015. \emph{Municipal Shared Services and Consolidation: A Public Solutions Handbook}. New York: Routledge.
\end{enumerate}

\subsection*{Additional Readings}

In addition to the above texts, several additional readings, including articles, book chapters, and case studies, are posted on Canvas and are noted in the course schedule at the end of this document. 


\section{University Student Policies}

In accordance with UPS 300.00, students must be familiar with certain policies applicable to all courses. Please review these policies as needed and visit this Cal State Fullerton website \href{https://t.ly/csuf-syllabus}{https://t.ly/csuf-syllabus} for links to the following information:

\begin{enumerate}
    \item   University learning goals and program learning outcomes.
    \item	Learning objectives for each General Education (GE) category.
    \item	Guidelines for appropriate online behavior (netiquette).
    \item	Students’ rights to accommodations for documented special needs.
    \item   Campus student support measures, including Counseling \& Psychological Services, Title IV and Gender Equity, Diversity Initiatives and Resource Centers, and Basic Needs Services.
    \item	Academic integrity (refer to UPS 300.021).
    \item	Actions to take during an emergency.
    \item	Library services information.
    \item	Student Information Technology Services, including details on technical competencies and resources required for all students.
    \item	Software privacy and accessibility statements.
\end{enumerate}

\section{Course Student Policies}

\subsection*{Course Communication}
All course announcements and communications will be sent via \emph{Canvas} and university email. Students are responsible for regularly checking their \emph{Canvas} notifications and email. Students are also responsible for ensuring that their \emph{Canvas} notifications are set to receive messages from the course. Students are expected to check \emph{Canvas} and their email at least once daily.

\subsection*{Due Dates}
If you have concerns about meeting assignment deadlines, please get in touch with the professor in advance to discuss potential accommodation.

\subsection*{Alternative Procedures for Submitting Work}
Students are expected to submit all assignments via \emph{Canvas}. If you cannot submit an assignment via \emph{Canvas}, please get in touch with the professor to discuss alternative submission procedures.

\subsection*{Extra Credit}
There are no extra credit assignments in this course. 

\subsection*{Academic Integrity}
Students are expected to adhere to the highest standards of academic integrity. Any student found to have engaged in academic dishonesty will be subject to the sanctions described in the \href{https://www.fullerton.edu/senate/publications_policies_resolutions/ups/UPS%20300/UPS%20300.021.pdf}{Academic Dishonesty Policy} (UPS 300.021). Academic dishonesty includes, but is not limited to, cheating, plagiarism, fabrication, facilitating academic dishonesty, and submitting previously graded work without prior authorization. Students are expected to be familiar with the university's policy on academic dishonesty and to adhere to this policy in all aspects of this course. Any student who has questions about the policy should ask the professor for clarification.

\section{Course Delivery}

This hybrid course blends online and in-person elements. In-person sessions are scheduled for August 30th and November 16th.

\subsection*{Flipped Classroom Model and Team-Based Learning}

Students should familiarize themselves with the course material before each session, as class time emphasizes discussions, group tasks, and collaborative activities.

\subsubsection*{Pre-class Preparation}

Students should, before each class:
    \begin{itemize}
        \item Review the assigned readings and course content;
        \item Watch the week's seminar instruction video;
        \item Take the weekly quiz, aiding the instructor in understanding points of clarity or confusion;
        \item Complete the weekly Annotated Bibliography and Synthesis assignment.
    \end{itemize}


\subsubsection*{In-Class Activities}

    Sessions on Zoom are on Tuesdays from 7:00 to 8:30 p.m., as detailed in the subsequent schedule. Activities include:

    \begin{itemize}
        \item A brief lecture on the week's theme;
        \item Team discussions rooted in the annotated bibliographies;
        \item Comprehensive class debates on the week's subject;
        \item Debriefing and dialogue about the synthesis assignment.
    \end{itemize}
    
\subsubsection*{Post-class Activities}

After each session, students should finish the weekly reflection assignment, promoting metacognitive practice and enabling introspection on their learning trajectory.

\section{Course Requirements}

\subsection{Participation}

Active participation, especially in online discussions, is essential. It accounts for 10\% of the final grade. Weekly, there will be one or two thematic discussions. Consistent involvement in these reflection discussions is anticipated.

\subsection{Intellectual Autobiography}

This 3-page essay should elucidate your intellectual evolution and its impact on your viewpoint regarding public administration. The assignment, worth 10\% of your total grade, is due by the end of week two. The following questions may help you get started:
    \begin{itemize}
        \item What experiences have shaped your intellectual journey?
        \item What are your intellectual interests?
        \item What are your career goals?
        \item How do you think this course will help you achieve your goals?
        \item What do you hope to learn from this course?
        \item What do you hope to contribute to this course?
    \end{itemize}
 

\subsection{Collaboration Simulation}

This group task emulates a collaborative governance procedure, accounting for 10\% of the final grade. It will occur during the second in-person session. Specific roles will be assigned, and further information will be shared during the initial in-person session.  

\subsection{Weekly Annotated Bibliography, Synthesis, and Reflection}

This assignment combines group and individual elements, starting with an annotated bibliography of the weekly readings and culminating in a 3-page synthesis. This component makes up 50\% of the final grade.

\subsubsection{Assignment Components and Submission Timelines}

    \begin{itemize}

    \item \textbf{Annotated Bibliography}: As a group task, it should be uploaded to the shared document \emph{two days before our class} each week.

    \item \textbf{Final Synthesis}: An individual task to be submitted on Canvas \emph{one day before our class} each week.

    \item \textbf{Reflection}: Submit individually on Canvas \emph{the day after our class}.
    
    \end{itemize}

\subsubsection{Details}

\begin{itemize}

    \item \textbf{Instructions}: Your weekly writing assignment begins with an annotated bibliography of each weekly reading to facilitate class discussion and prepare you for comprehensive exams. It concludes with a 3-page synthesis of the week’s readings. This assignment incorporates a group activity using an AI large language model (e.g., ChatGPT) and individual work.

    \item \textbf{Objectives}: To engage deeply with the course readings, reflect on their implications for public administration, and collaborate with peers and AI to continuously improve analytical and writing skills.

    \item \textbf{Details}: We’re incorporating AI (ChatGPT) to facilitate assignment feedback, aiming to refine your comprehension and writing skills. While ChatGPT is a powerful tool, remember that it is designed to complement your critical thinking, not replace it.
    
\end{itemize}

\subsubsection{An Assignment in Three Parts}

\begin{enumerate}

\item \textbf{Annotated Blibliography}: 

\begin{enumerate}
    \item \textbf{Purpose}: Review literature pertinent to the week's theme, facilitating comprehensive discussions and in-depth syntheses. This part of the assignment has an individual and a group component.
    \item \textbf{Process}: You will write an annotated bibliography of the week's readings each week. The annotated bibliography should be 150 words long, written in your own words, and include a citation for each source.  The annotation should summarize the source's central theme and main points and note its relevance to the week's topic.
    \begin{enumerate}
        \item The class will be divided into teams each week, and you will create a common document—a shared Word, Google Doc, or similar—to share your annotated bibliographies. This document will be used to facilitate class discussion.
        \item \textbf{Optional Activity}: Submit your annotated bibliography to the AI for feedback and revise your annotated bibliography based on the AI's feedback.
            \begin{itemize}
                \item Use this prompt: 
                \textit{"I've completed an annotated bibliography for an article. The citation is [insert citation]. The summary I wrote is [insert summary]. I've noted the relevance as [insert relevance]. Could you provide feedback on my summary and relevance note?"}
                \item This optional activity is for your benefit only and will not be graded. You may submit as many annotated bibliographies as you wish for feedback. 
            \end{itemize}
        \item \textbf{Note}: This portion of the assignment will be graded as a group assignment. By the due date each week, there should be \textit{at least} one annotated bibliography for each reading. The group will receive a single grade for the week's annotated bibliographies. If the group fails to submit at least one annotated bibliography for each reading, the group will receive a zero for the week's assignment.
    \end{enumerate}
\end{enumerate}

\item \textbf{Synthesis}:

\begin{enumerate}
    \item \textbf{Purpose}: Pull together the main themes and insights from the week’s readings and discuss their implications for public administration. This part of the assignment has an individual and an AI review component.
    \item \textbf{Process}: You will write a 3-page synthesis of the week's readings each week. The synthesis should not be a simple summary of each source but rather a critical analysis that identifies patterns, draws connections, and addresses contradictions among the sources. The synthesis encourages you to think critically and analytically about the readings and their implications for the theory and practice of public administration.
        \begin{itemize}
            \item Upon completion, submit it to the AI for feedback. Use this prompt:  \textit{“I’ve synthesized information from [number] articles on [topic]. Here’s my draft of the synthesis [insert draft]. Could you provide feedback and suggest any connections or contrasts I missed?”}
            \item Revise your synthesis based on the AI's feedback.
            \item Submit your revised synthesis \emph{and} the AI's feedback to Canvas for grading.
        \end{itemize}
    \item A second note of caution: ChatGPT is a powerful tool but not a substitute for critical thinking. The AI will not be able to provide feedback on the quality of your analysis or the accuracy of your understanding. It will only be able to provide feedback on the clarity of your writing and the coherence of your argument.    
    \item Your revised synthesis will be scored using the rubric in Table~\ref{tab:synthesis-rubric} at the end of this document. 
\end{enumerate}

\item \textbf{Refection Activity}
    \begin{enumerate}
        \item \textbf{Purpose}: To engage in metacognitive practice, allowing you to analyze your learning process, the feedback you've received, and the evolution of your understanding of the week's topic. 
        \item \textbf{Process}: After watching the lecture, attending class, completing your revised synthesis, and reviewing the feedback from ChatGPT, take some time to reflect on the entire process, including the annotated bibliographies portion of the assignment. 
        \item Consider any or all of the following questions:
        \begin{itemize}
            \item How did your understanding of the topics evolve?
            \item How did the group process impact your perspective on the topic or your approach to writing the synthesis?
            \item What strategies did you find most effective in distilling and presenting information coherently?
            \item How can you apply the lessons from this assignment to future tasks in public administration or other professional settings?
            \item What did you gain from the weekly lecture and class discussion?
            \item What did you learn from your peers in the weekly discussion?
        \end{itemize}
        \item Write a brief reflection encapsulating your thoughts.
        \item Share this reflection with ChatGPT for feedback using the prompt: \textit{"I've completed a reflection on my learning process for this week's synthesis assignment. Here's my reflection [insert reflection]. Could you provide feedback or ask questions to provoke further thought?”}
        \item Revise your reflection after receiving feedback from the AI.
        \item Share your revised reflection with the class on the discussion board.
    \end{enumerate}
\end{enumerate}

\subsubsection{Skip Week}
\textbf{Purpose}: To provide a break from the weekly writing assignment and allow you to focus on other assignments and responsibilities. You can skip the assignment one time without penalty. Please notify the instructor in advance if you plan to skip the assignment.

\subsubsection{Grading}
\begin{itemize}
    \item Annotated Bibliography (Group): 20\%
    \item Synthesis (Individual): 70\% (Comprising Completeness, Quality, and AI Feedback components)
    \item Discussion Board Reflection: 10\%
\end{itemize}

\subsection{Final Course Reflection}

A final course reflection is due in the last week of the semester. This reflection should be 5–7 pages long and reflect on your course learning experience. This assignment is worth 15\% of your total grade. 

Reflect on your journey through the course, weaving together your intellectual autobiography, the collaboration simulation, the weekly assignments, and insights gained from course materials and discussions.

\subsection*{Guidelines}

\begin{enumerate}
    \item Introduction: provide an overview of your initial expectations entering the course and the major assignments and activities you completed.
    \item Intellectual Autobiography Revisited: 
        \begin{itemize}
            \item Reflect on the intellectual autobiography you wrote at the beginning of the course.
            \item How has your intellectual journey evolved over the semester?
            \item Have any of your goals or interests shifted? If so, how and why?
        \end{itemize}
    \item Collaboration Simulation Experience:
        \begin{itemize}
            \item Discuss your role in the collaboration simulation and the group dynamics you observed.
            \item How did the experience align or differ from the theories and concepts discussed in the course?
            \item Reflect on the challenges and benefits of collaborative governance based on your simulation experience.
        \end{itemize}

    \item Weekly Assignments:
        \begin{itemize}
            \item Synthesize your weekly reflections. Highlight any recurring themes, challenges, or breakthrough moments you had.
            \item How did the course materials and weekly assignments inform or challenge your perspectives on public administration?
            \item What did you learn from your peers in the weekly discussions?
        \end{itemize}

    \item Personal Growth and Future Implications:
        \begin{itemize}
            \item How has your understanding of public administration evolved over the semester?
            \item How will you apply the lessons from this course to your future career?
            \item What are your next steps in your intellectual journey?
        \end{itemize}

    \item Conclusion: Provide a summary of your reflections and insights. Offer any feedback or suggestions for improving future iterations of the course.
    
\end{enumerate}

\section{Grades}

\subsection*{Grading Scale and Grade Weights}  
The grading scale is shown in Table~\ref{tab:grading-scale}. Grades will be given based on Table~\ref{tab:grade-weights} weights.

\begin{table}[ht]
\centering
\caption{Grading Scale}
\begin{tabular}{llll}
\toprule
\textbf{Grade} & \textbf{Percentage} & \textbf{Grade} & \textbf{Percentage} \\
\midrule
A+ & 98.0 – 100 & B- & 80.0 – 81.9\\
A & 92.0 – 97.9 & C+ & 78.0 – 79.9\\
A- & 90.0 – 91.9 & C & 72.0 – 77.9\\
B+ & 88.0 – 89.9 & C- & 70.0 – 71.9\\
B & 82.0 – 87.9 & & \\
\bottomrule
\end{tabular}
\label{tab:grading-scale}
\end{table}

\begin{table}[ht]
    \centering
    \caption{Grade Weights}
    \begin{tabular}{ll}
        \toprule
    \textbf{Assignment} & \textbf{Percentage} \\
    \midrule
    Participation & 10\% \\
    Intellectual Autobiography & 10\% \\
    Collaboration Simulation & 15\% \\
    Weekly Annotated Bibliography, Synthesis, and Reflection & 50\% \\
    Final Course Reflection & 15\% \\
    \bottomrule
    \end{tabular}
    \label{tab:grade-weights}
    \end{table}

\newpage

\section{Course Schedule}

\begin{itemize}
	\item[] \textbf{Week 1 (August 27): Introduction and Welcome}
	\begin{itemize}
		\item IN-PERSON Introduction, Overview, and Expectations
		\item Syllabus Review
	\end{itemize} 
	
	\item[] \textbf{Week 2 (September 3): Foundations and Key Topics}
		\begin{itemize}
			\item \citet[chap.~1]{Henderson2015}
			\item \citet[chap.~1]{Agranoff2012}
			\item \citet{Bryson2006} 
			\item \citet[chap.~1]{Bingham2008}
			\item INTELLECTUAL AUTOBIOGRAPHY DUE
		\end{itemize}
	
	
	\item[] \textbf{Week 3 (September 10): Boundaries, Federalism, and Intergovernmental Relations}
		\begin{itemize}
			\item \citet[chap.~2–3]{Agranoff2012}
			\item  \citet[chap.~intro,~1]{Agranoff2017}
			\item  \citet{Schneider2009}
			\item \cite{Gerlak2006}
		\end{itemize}
		
	\item[] \textbf{Week 4 (September 17): The Costs of Service Cooperation; Polycentricity}
		\begin{itemize}
			\item \citet[chap.~2]{Henderson2015}
			\item  \citet[chap.~1]{Oakerson1999}
			\item  TBD
			%\item  TBD
		\end{itemize}
	
	\item[] \textbf{Week 5 (September 24): Communities and Culture}
		\begin{itemize}
			\item \citet[chap.~3]{Henderson2015}
			\item  \citet[chap.~4]{Wondolleck2000}
			\item  \citet[chap.~1]{Sirianni2009}
			%\item  TBD
		\end{itemize}

	
	\item[] \textbf{Week 6 (October 1): Meet with Collaboration Simulation Facilitator}
		
	
	\item[] \textbf{Week 7 (October 8): Consolidation, Contracts, External Agreements}
		\begin{itemize}
			\item \citet[chaps.~4–5]{Agranoff2012}
			\item \citet[chaps.~4–6]{Henderson2015}
			\item  \citet[chap.~10]{Bingham2008}
			\item  \citet[chap.~7]{Feiock2010}
		\end{itemize}
	
	\item[] \textbf{Week 8 (October 15): Collaboration in Practice}
		\begin{itemize}
            \item \citet[chap.~6]{Henderson2015}
            \item \citet[chap.~6]{Agranoff2012}			
		\end{itemize}
		
	
	\item[] \textbf{Week 9 (October 22): Managing in Networks}
		\begin{itemize}
			\item \citet[chap.~6]{Agranoff2012}
			\item \citet[chap.~3]{Kickert1997}
			\item  \citet[chap.~1]{Goldsmith2009}
			\item  \citet[chap.~9]{Bingham2008}
			\item  \citet[chap.~8]{Agranoff2017}
		\end{itemize}

    \item[] \textbf{Week 9 (October 29): Collaboration Simulation}
        \begin{itemize}
            \item IN-PERSON Collaboration Simulation
        \end{itemize}
		
	
	\item[] \textbf{Week 11 (November 5): Movie Night}
		\begin{itemize}
			\item A link to a movie will be placed on Canvas. You will have one week to watch the movie and complete a short reflection.
			\item \emph{The instructor will attend a conference this week.}
		\end{itemize}
	
	\item[] \textbf{Week 12 (November 12): Barriers to Collaboration}
		\begin{itemize}
			\item \citet[chap.~7]{Agranoff2012}
			\item \citet[chap.~8]{Henderson2015}
			\item  \citet[chaps.~3–4]{OLeary2009}
			\item  \citet{OToole2004}
		\end{itemize}
	
	\item[] \textbf{Week 13 (November 19): New Organizations and Local Public Management}
		\begin{itemize}
			\item \citet[chap.~8]{Agranoff2012}
			\item \citet[chaps.~7,~9]{Henderson2015}
			\item  \citet[chaps.~5–6]{OLeary2009}
			\item  \citet{Williams2016}
		\end{itemize}

	
	\item[] \textbf{Week 14 (December 3): Advancing Collaboration Theory}
		\begin{itemize}
			\item \citet{Williams2016a}
			\item  \citet{Dietz2003}
			\item  \citet{Lubell2007}
			\item  \citet{Lubell2010}

		\end{itemize}
	
	\item[] \textbf{Week 15 (December 10): Conclusion and The Future}
		\begin{itemize}
			\item \citet[chap.~9]{Agranoff2012}
			\item \citet[chap.~10]{Henderson2015}
			\item  \citet[chap.~9]{Agranoff2017}
			\item  \citet[chap.~14]{Bingham2008}
			\item  \cite[chap.~10]{Kickert1997}
		\end{itemize}
	
	\item[] \textbf{Week 16 (December 17): Final Reflections Due}
	
\end{itemize}

\bibliographystyle{apsr}
\bibliography{588}



\appendix

\begin{table}[ht]
    \footnotesize
    \centering
    \caption{Synthesis Grading Rubric}
    \begin{tabularx}{\textwidth}{|X|X|X|X|X|X|}
    \toprule
    \textbf{Criteria} & \textbf{Exemplary (5)} & \textbf{Proficient (4)} & \textbf{Satisfactory (3)} & \textbf{Developing (2)} & \textbf{Beginning (1)} \\
    \midrule
    \textbf{Understanding of Readings} & Demonstrates deep understanding and insightful interpretation of all readings. & Shows a solid understanding and interpretation of readings. Few minor omissions. & Shows a basic understanding of most readings but has some significant omissions or misinterpretations. & Limited understanding of the readings. Frequent omissions or misinterpretations. & Demonstrates minimal or no understanding of the readings. \\
    \midrule
    \textbf{Identification of Themes \& Patterns} & Clearly and comprehensively identify all major themes and patterns across the readings. & Identifies most major themes and patterns. A few minor omissions. & Identifies some themes, but misses major patterns or connections. & Struggles to identify themes and patterns. Many omissions. & Fails to identify or connect any themes and patterns. \\
    \midrule
    \textbf{Analysis \& Critical Thinking} & Offers a deep and nuanced analysis, draws sophisticated connections, and critically addresses contradictions. & Provides a solid analysis, draws relevant connections, and addresses some contradictions. & Some analysis and connection, but lacks depth and may not address contradictions. & Limited analysis. Draws few connections. Many missed opportunities for deeper thinking. & Lacks any real analysis. No connections made. \\
    \midrule
    \textbf{Structure \& Organization} & Ideas are organized in a logical, coherent manner. The flow enhances the argument and understanding. & Ideas are mostly well organized. A few minor structural issues but they don't majorly impede understanding. & Organization is somewhat clear, but reader may get lost at times. Some areas lack coherence. & Ideas and sections are disorganized. Hard for the reader to follow. & Lacks any clear structure or organization. Very hard to follow. \\
    \midrule
    \textbf{Clarity \& Language} & Writing is clear, concise, and jargon-free. No grammatical or stylistic errors. & Writing is mostly clear with minor stylistic or grammatical issues. Might have occasional jargon. & Writing is somewhat clear but can be verbose or have some jargon. Several grammatical issues. & Writing is often unclear, with frequent use of jargon and many grammatical issues. & Writing is consistently unclear. Pervasive jargon and grammatical errors. \\
    \bottomrule
    \end{tabularx}
    \label{tab:synthesis-rubric}
    \end{table}



\end{document}