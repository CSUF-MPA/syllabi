\documentclass[12pt, letterpaper]{article}
\usepackage[english]{babel}
\usepackage[T1]{fontenc}
%\usepackage[light]{ubuntu}
\usepackage[margin=1.25in]{geometry}
\usepackage{xcolor}
\usepackage{url}
\usepackage[utf8]{inputenc}
\usepackage{tabularx}
\usepackage{booktabs}
\frenchspacing
\usepackage{multicol}
\usepackage{eso-pic}
\usepackage[longnamesfirst]{natbib}
\bibpunct{(}{)}{;}{a}{}{,}
\usepackage{caption}
\usepackage{subcaption}
\usepackage{setspace}
\usepackage{paralist}
\usepackage{quoting}
\usepackage{comment}
\usepackage{enumitem}
\usepackage{graphicx}
\usepackage{float}
\usepackage{bookmark}
\renewcommand{\thesection}{\arabic{section}.}
\renewcommand{\thesubsection}{\thesection\arabic{subsection}}
\renewcommand{\thesubsubsection}{\thesubsection.\arabic{subsubsection}}
\usepackage{hyperref}
\hypersetup{
    pdftitle={Syllabus: POSC521 Fall 2023}, 
    pdfauthor={David P. Adams, Ph.D.},
    pdfsubject={MPA Capstone Seminar Syllabus},
    pdfkeywords={syllabus, 521, MPA, Adams, CSUF, POSC, public administration, theory, literature review, AI, ChatGPT},
    pdfproducer={VS Code 1.60.2}, 
    pdfcreator={pdflatex}
}
\usepackage{fancyhdr}
\pagestyle{fancy}
\fancyhf{}
\lhead{\textit{POSC 521}}
\rhead{\textit{Fall 2023}}
\lfoot{\textit{David P. Adams, Ph.D.}}
\rfoot{\textit{Page \thepage}}
\renewcommand{\headrulewidth}{0pt}
\renewcommand{\footrulewidth}{0pt}
\usepackage{titlesec}
\titleformat{\section}{\normalfont\fontsize{14}{15}\bfseries}{\thesection}{1em}{}
\titleformat{\subsection}{\normalfont\fontsize{12}{15}\bfseries}{\thesubsection}{1em}{}
\titleformat{\subsubsection}{\normalfont\fontsize{12}{15}\bfseries}{\thesubsubsection}{1em}{}
\titleformat{\paragraph}{\normalfont\fontsize{12}{15}\bfseries}{\theparagraph}{1em}{}
\titleformat{\subparagraph}{\normalfont\fontsize{12}{15}\bfseries}{\thesubparagraph}{1em}{}
\titlespacing*{\section}{0pt}{0.5\baselineskip}{0.5\baselineskip}
\titlespacing*{\subsection}{0pt}{0.5\baselineskip}{0.5\baselineskip}
\titlespacing*{\subsubsection}{0pt}{0.5\baselineskip}{0.5\baselineskip}
\titlespacing*{\paragraph}{0pt}{0.5\baselineskip}{0.5\baselineskip}
\titlespacing*{\subparagraph}{0pt}{0.5\baselineskip}{0.5\baselineskip}




\begin{document}
\title{MPA Capstone Seminar: \\ Public Administration Theory}

\author{POSC 521 -- Fall 2023}
\date{Thursdays at 7:00 in Humanities 226}

    \maketitle


\subsection*{Professor: David P. Adams, Ph.D.}

\subsubsection*{Contact Information:}

\begin{itemize}
	\item Office: 516 Gordon Hall
	\item Phone/SMS: (657) 278-4770
	\item website: \href{https://dadams.site}{\texttt{https://dadams.site}}
	\item email: \href{dpadams@fullerton.edu}{\texttt{dpadams@fullerton.edu}}
	\item Office hours: Tuesdays \& Thursdays from 9:30 to 11:00, Thursdays from 5:30 to 6:30, and by \href{https://t.ly/dpa-appt}{appointment}.
	\item Schedule meetings throughout the week: \href{https://t.ly/dpa-appt}{\texttt{https://t.ly/dpa-appt}}
\end{itemize}

\section{Catalog Description}
Concepts, models and ideologies of public administration within the larger political system. Course restricted to students in their final six units of graduate work.

\section{Course Description}
The capstone seminar in the Master of Public Administration program at Cal State Fullerton examines concepts, models, and ideologies of public administration within the larger political system.

\section{Course Objectives}
This course is designed to accomplish five interrelated objectives:

\begin{enumerate}
    \item \textbf{Theory Examination}: We will delve into the most important theories and literature in public administration, fostering a deep understanding of the field.
    \item \textbf{Literature Review}: You will complete a literature review in your concentration area, allowing you to specialize and delve deeper into a specific aspect of public administration. This preparation will be crucial for the general concentration portion of the comprehensive exams.
    \item \textbf{Writing Skills}: This course will enhance your writing skills, focusing on clear, concise, and effective communication. This preparation will be crucial for the general theory portion of the comprehensive exams.
    \item \textbf{Peer Review and Collaboration}: Through a structured peer review process, you will learn to provide constructive feedback, gain new perspectives, and improve your own work based on your peers’ insights. This collaborative learning approach is designed to mimic real-world public administration environments where collaboration and feedback are key.
    \item \textbf{AI-Assisted Learning}: We will be incorporating AI (ChatGPT) into our learning process. You will interact with this AI to refine your understanding of weekly readings and improve your assignments. This innovative approach enhances your learning experience and prepares you for a future where AI tools are increasingly prevalent in public administration.
\end{enumerate}

\subsection*{Course Topics}
Throughout the course, we will cover the following topics:
\begin{table}[h]
    \centering
    \caption{Public Administration Topics}
    \begin{tabular}{ll}\hline
        Public Administration Theory & Public Administration in the U.S. Context \\
        Street-Level Bureaucrats & Public Policy and Implementation \\
        Public Service Values and Ethics & Privatization and Contracting \\
        Leadership and Management & Performance Management \\
        Motivation & Nonprofits and Public Administration \\
        The Future of Public Administration & Current Issues in Public Administration \\ \hline
    \end{tabular}
    \label{tab:pa_topics}
    \end{table}

\section{Course Materials}
\subsection*{Required Texts}
\begin{itemize}
    \item \textbf{Denhardt and Denhardt}. \textit{The New Public Service: Serving, Not Steering}. 4th ed. Routledge, 2015.
    \item \textbf{Gooden, Susan T.}. \textit{Race and Social Equity: A Nervous Area of Government}. Oxford University Press, 2014.
    \item \textbf{Kamarck, Elaine C.}. \textit{The End of Government…As We Know It: Making Public Policy Work}. 2nd ed. Lynne Rienner Publishers, 2015.
    \item \textbf{Lipsky, Michael}. \textit{Street-Level Bureaucracy: Dilemmas of the Individual in Public Services}. Russell Sage Foundation, 2010.
\end{itemize}

\begin{itemize}
    \item \cite{Denhardt2015} develop a framework emphasizing the importance of public service values, democratic engagement, and collaboration between citizens and government. The authors argue that the New Public Service (NPS) is a departure from the traditional public administration model, which prioritizes efficiency and effectiveness and instead emphasizes the public interest, social equity, and the need for public servants to engage with citizens and communities. The NPS emphasizes that public service is a calling and requires a commitment to serving the public good above personal gain or profit. The framework has implications for how public organizations are structured, how they engage with citizens, and how they are held accountable for their actions.
    \item \cite{Gooden2014} explores the historical and contemporary challenges surrounding race and social equity within public administration. The book highlights the importance of addressing racial disparities in policy outcomes and emphasizes the critical role of public administrators in promoting social equity. Through a comprehensive analysis of the intersection between race and public policy, Gooden offers practical strategies and best practices for fostering a diverse and inclusive public service workforce. This essential resource equips students and practitioners with the knowledge and tools necessary to confront racial disparities and work towards a more equitable society in their roles as public administrators.
    \item \cite{Kamarck2007} posits that the traditional bureaucratic model of government is evolving into a more efficient and responsive system due to the transformative power of technology and globalization. She contends that this shift is characterized by the rise of networked governance, in which governments increasingly collaborate with private and non-governmental organizations to deliver public services and adopt performance-based management techniques. While challenging the status quo, Kamarck argues that these changes enhance governments’ capacity to address complex social and economic issues, providing citizens with more effective and accountable institutions.
    \item \cite{Lipsky2010} focuses on the role of front-line public service workers, or ``street-level bureaucrats,'' who directly interact with the public and implement policies. He explores the dilemmas these workers face, such as limited resources, conflicting goals, and the need to exercise discretion in decision-making. The book relates to public administration by illuminating the importance of understanding the experiences and challenges faced by street-level bureaucrats as their actions determine the success or failure of public policies.
\end{itemize}

\subsection*{Additional Readings are indicated in the course schedule below.}

\section{University Student Policies}

In accordance with UPS 300.00, students must be familiar with certain policies applicable to all courses. Please review these policies as needed and visit this Cal State Fullerton website \href{https://t.ly/csuf-syllabus}{https://t.ly/csuf-syllabus} for links to the following information:

\begin{enumerate}
    \item   University learning goals and program learning outcomes.
    \item	Learning objectives for each General Education (GE) category.
    \item	Guidelines for appropriate online behavior (netiquette).
    \item	Students’ rights to accommodations for documented special needs.
    \item   Campus student support measures, including Counseling \& Psychological Services, Title IV and Gender Equity, Diversity Initiatives and Resource Centers, and Basic Needs Services.
    \item	Academic integrity (refer to UPS 300.021).
    \item	Actions to take during an emergency.
    \item	Library services information.
    \item	Student Information Technology Services, including details on technical competencies and resources required for all students.
    \item	Software privacy and accessibility statements.
\end{enumerate}

\section{Course Student Policies}

\subsection*{Course Communication}
All course announcements and communications will be sent via \emph{Canvas} and university email. Students are responsible for regularly checking their \emph{Canvas} notifications and email. Students are also responsible for ensuring that their \emph{Canvas} notifications are set to receive messages from the course. Students are expected to check \emph{Canvas} and their email at least once daily.

\textbf{Response Time}: I will strive to respond to all student emails and \emph{Canvas} messages within 24 hours, except on weekends and holidays. If you do not receive a response within 24 hours, please send a follow-up message. If you do not receive a response within 48 hours, please send another follow-up message and contact me via phone or SMS text at (657) 278-4770.

\subsection*{Due Dates}
Please know that exams are only permitted on the scheduled date as indicated in the course schedule below. If you have concerns about meeting assignment deadlines, please contact the professor in advance to discuss potential accommodation.

\subsection*{Alternative Procedures for Submitting Work}
Students are expected to submit all assignments via \emph{Canvas}. If you cannot submit an assignment via \emph{Canvas}, please contact the professor to discuss alternative submission procedures.

\subsection*{Extra Credit}
An extra credit opportunity exists for each of the weekly synthesis assignments. Students can receive up to an additional five points for completing a reflection on the assignment using the criteria indicated below. Please do not ask for additional extra credit assignments.

\subsection*{Academic Integrity}
Students are expected to adhere to the highest standards of academic integrity. Any student found to have engaged in academic dishonesty will be subject to the sanctions described in the \href{https://www.fullerton.edu/senate/publications_policies_resolutions/ups/UPS%20300/UPS%20300.021.pdf}{Academic Dishonesty Policy} (UPS 300.021). Academic dishonesty includes, but is not limited to, cheating, plagiarism, fabrication, facilitating academic dishonesty, and submitting previously graded work without prior authorization. Students are expected to be familiar with the university's policy on academic dishonesty and to adhere to this policy in all aspects of this course. Any student who has questions about the policy should ask the professor for clarification.


\section{Course Requirements}

\subsection{Weekly Annotated Bibliography, Synthesis, and Reflection}

To facilitate class discussion and prepare you for comprehensive exams, your weekly writing assignment begins with an annotated bibliography of each weekly reading. It concludes with a 2-page synthesis of the week’s readings. This assignment incorporates a group activity using an AI large language model (ChatGPT) and individual work.

\subsubsection*{Definitions}

\textbf{Annotated Bibliography}: An annotated bibliography is an organized list of sources (like a reference list), where each source is followed by a brief (usually about 150 words) descriptive and evaluative paragraph—the annotation. In this course, each annotation should consist of a full citation, a summary of the central theme and key points of the source, and a note on its relevance to the week’s topic.

The annotated bibliography in this course is to review the literature on the week’s topic and prepare you for in-depth discussion and analysis. It also helps develop your ability to distill and synthesize complex information, a key skill in public administration.

\textbf{Synthesis}: A synthesis is a piece of writing that combines information from many sources to present an understanding or make a point. The sources used in a synthesis may be different articles, essays, interviews, or lectures. In this case, your sources are the readings for the week.

The purpose of the synthesis in this course is to pull together the main themes and insights from the week’s readings and discuss their implications for public administration. The synthesis should not be a simple summary of each source but rather a critical analysis that identifies patterns, draws connections, and addresses contradictions among the sources. The synthesis encourages you to think critically and analytically about the readings and their implications for the theory and practice of public administration.

\subsubsection*{Details}

\textbf{Instructions}: Your weekly writing assignment begins with an annotated bibliography of each weekly reading to facilitate class discussion and prepare you for comprehensive exams. It concludes with a 2-page synthesis of the week’s readings. This assignment incorporates a group activity, the use of an AI large language model (e.g., ChatGPT), and individual work.

\textbf{Objectives}: To engage deeply with the course readings, reflect on their implications for public administration, and collaborate with peers and AI to continuously improve analytical and writing skills.

\textbf{Details}: We’re incorporating AI (ChatGPT) to facilitate feedback on your assignments, aiming to refine your comprehension and writing skills. While ChatGPT is a powerful tool, always remember that it is designed to complement, not replace, your own critical thinking.

\subsubsection*{An Assignment in Two (or Three) Parts}

\textbf{Annotated Blibliography}: 

\begin{enumerate}
    \item \textbf{Purpose}: Review the literature on the week’s topic, preparing you for an in-depth synthesis and class discussion. This part of the assignment has an individual and a group component.
    \item \textbf{Process}: You will write an annotated bibliography of the week's readings each week. The annotated bibliography should be 150 words in length, written in your own words, and include a citation for each source.  The annotation should summarize the source's central theme and main points and note its relevance to the week's topic.
    \begin{enumerate}
        \item The class will be divided into groups each week, and you will create a common document—a shared Word, GoogleDoc, or similar—to share your annotated bibliographies. This document will be used to facilitate class discussion and prepare for comprehensive exams.
        \item \textbf{Optional Activity}: Submit your annotated bibliography to the AI for feedback and revise your annotated bibliography based on the AI's feedback.
            \begin{itemize}
                \item Use this prompt: 
                \textit{"I've completed an annotated bibliography for an article. The citation is [insert citation]. The summary I wrote is [insert summary]. I've noted the relevance as [insert relevance]. Could you provide feedback on my summary and relevance note?"}
                \item This optional activity is for your benefit only and will not be graded. You may submit as many annotated bibliographies as you wish for feedback. 
            \end{itemize}
        \item \textbf{Note}: This portion of the assignment will be graded as a group assignment. By the due date each week, there should be \textit{at least} one annotated bibliography for each reading. The group will receive a single grade for the week's annotated bibliographies. If the group fails to submit at least one annotated bibliography for each reading, the group will receive a zero for the week's assignment.
    \end{enumerate}
\end{enumerate}

\textbf{Synthesis}:

\begin{enumerate}
    \item \textbf{Purpose}: Pull together the main themes and insights from the week’s readings and discuss their implications for public administration. This part of the assignment has an individual and a peer review component.
    \item \textbf{Process}: You will write a 2-page synthesis of the week's readings each week. The synthesis should not be a simple summary of each source but rather a critical analysis that identifies patterns, draws connections, and addresses contradictions among the sources. The synthesis encourages you to think critically and analytically about the readings and their implications for the theory and practice of public administration.
        \begin{itemize}
            \item Upon completion, submit your draft to Canvas for peer review. Canvas will automatically assign one peer to review for each student. Read your peer's synthesis and use the rubric in Table~\ref{tab:peer-review-rubric} at the end of this document. The rubric is only a guide, so feel free to provide additional feedback as you see fit, and don't feel obligated to comment on every criterion. Be kind, constructive, and specific in your feedback. 
            \item Revise your synthesis based on the peer review and submit it to the AI for feedback. Use this prompt:  \textit{“I’ve synthesized information from [number] articles on [topic]. Here’s my draft of the synthesis [insert draft]. Could you provide feedback and suggest any connections or contrasts I missed?”}
            \item Submit your revised synthesis and the AI's feedback to Canvas for grading.
        \end{itemize}
    \item \textbf{Grading}: The synthesis will be graded based on the rubric below. The peer review will be graded based on the quality of the feedback provided. The AI feedback will be graded based on the quality of the feedback provided. The synthesis, peer review, and AI feedback will be graded as a single assignment according to the following weights:
        \begin{itemize}
            \item Completeness: 20\%
            \item Quality: 40\%
            \item Peer Review: 10\%
            \item AI Feedback: 10\%
        \end{itemize}
    Your revised synthesis will be scored using the rubric in Table~\ref{tab:synthesis-rubric} at the end of this document. 
\end{enumerate}

\textbf{Bonus Point Refection Activity}
    \begin{enumerate}
        \item \textbf{Purpose}: To engage in metacognitive practice, allowing you to analyze your learning process, the feedback you've received, and the evolution of your understanding of the week's topic. This optional reflection is individual and will be graded as a bonus point assignment.
        \item \textbf{Process}: After completing your revised synthesis and reviewing the feedback from ChatGPT, take some time to reflect on the entire process, including the annotated bibliographies and peer review portions of the assignment. 
        \item Consider none, any, or all of the following questions:
        \begin{itemize}
            \item How did your understanding of the readings evolve?
            \item Which feedback (from peers or ChatGPT) was most beneficial, and why?
            \item How did the peer review process impact your perspective on the topic or your approach to writing the synthesis?
            \item What strategies did you find most effective in distilling and presenting information coherently?
            \item How can you apply the lessons from this assignment to future tasks in public administration or other professional settings?
        \end{itemize}
        \item Write a brief reflection encapsulating your thoughts.
        \item Share this reflection with ChatGPT for feedback using the prompt: \textit{"I've completed a reflection on my learning process for this week's synthesis assignment. Here's my reflection [insert reflection]. Could you provide feedback or ask questions to provoke further thought?”}
        \item Submit your reflection and a copy of the AI's feedback to Canvas for bonus point grading.
    \end{enumerate}


\subsubsection*{Skip Week}
\textbf{Purpose}: To provide a break from the weekly writing assignment and allow you to focus on other assignments and responsibilities. You can skip the assignment one time without penalty.
    \begin{enumerate}
        \item You may skip one week of the annotated bibliography and synthesis assignment without penalty.
        \item You must notify the instructor by email before the start of the week you wish to skip.
    \end{enumerate}

\subsubsection*{Grading}
    \begin{enumerate}
        \item \textbf{Annotated Bibliography (Group)}: 20\% of the weekly grade
        \item \textbf{Synthesis (Individual)}: 80\% of the weekly grade
            \begin{itemize}
                \item Completeness: 20\%
                \item Quality: 40\%
                \item Peer Review: 10\%
                \item AI Feedback: 10\%
            \end{itemize}
        \item \textbf{Bonus Point Reflection (Individual)}: 5 possible points per week
    \end{enumerate} 


\begin{table}[h]
\tiny
\centering
\caption{Peer Review Evaluation Criteria}
\begin{tabularx}{\textwidth}{|X|X|X|X|}
\toprule
\textbf{Criteria} & \textbf{Excellent (5)} & \textbf{Good (4)} & \textbf{Needs Improvement (1-3)} \\
\midrule
\textbf{Clarity \& Organization} & The synthesis is exceptionally clear, well-organized, and easy to follow. Logical flow is evident, guiding the reader smoothly through the text. & The synthesis is clear and mostly organized. & The synthesis lacks clarity in some sections, and the organization needs refinement. It can be hard to follow or understand the author's main points in places. \\
\midrule
\textbf{Coverage of Key Themes} & All key themes from the readings are addressed comprehensively and accurately. Deep insights are evident. & Most key themes are addressed adequately. & Some major themes are overlooked or misrepresented. The synthesis might be too superficial or miss the deeper implications of the readings. \\
\midrule
\textbf{Critical Analysis} & The synthesis demonstrates deep critical analysis, highlighting patterns, drawing connections, and addressing contradictions among sources effectively. & Adequate critical analysis is evident. & Critical analysis is lacking. The synthesis might lean towards summary rather than analysis, or it might not sufficiently address patterns or contradictions. \\
\midrule
\textbf{Relevance to Public Administration} & The implications for public administration are clearly and effectively drawn out, adding value to the discussion. & Some relevance to public administration is noted. & The connection to public administration is weak, unclear, or not evident. The synthesis might benefit from more explicit ties to the field's theories and practices. \\
\midrule
\textbf{Constructive Feedback} & Feedback is specific, actionable, and supportive, clearly aimed at helping the author improve their synthesis. & Feedback is generally helpful. & Feedback is vague, overly general, or not constructive. It might not provide the author with clear steps or insights on how to improve. \\
\bottomrule
\end{tabularx}
\label{tab:peer-review-rubric}
\end{table}



\begin{table}[h!]
\tiny
\centering
\caption{Synthesis Grading Rubric}
\begin{tabularx}{\textwidth}{|X|X|X|X|X|X|}
\toprule
\textbf{Criteria} & \textbf{Exemplary (5)} & \textbf{Proficient (4)} & \textbf{Satisfactory (3)} & \textbf{Developing (2)} & \textbf{Beginning (1)} \\
\midrule
\textbf{Understanding of Readings} & Demonstrates deep understanding and insightful interpretation of all readings. & Shows a solid understanding and interpretation of readings. Few minor omissions. & Shows a basic understanding of most readings but has some significant omissions or misinterpretations. & Limited understanding of the readings. Frequent omissions or misinterpretations. & Demonstrates minimal or no understanding of the readings. \\
\midrule
\textbf{Identification of Themes \& Patterns} & Clearly and comprehensively identifies all major themes and patterns across the readings. & Identifies most major themes and patterns. A few minor omissions. & Identifies some themes, but misses major patterns or connections. & Struggles to identify themes and patterns. Many omissions. & Fails to identify or connect any themes and patterns. \\
\midrule
\textbf{Analysis \& Critical Thinking} & Offers a deep and nuanced analysis, draws sophisticated connections, and critically addresses contradictions. & Provides a solid analysis, draws relevant connections, and addresses some contradictions. & Some analysis and connection, but lacks depth and may not address contradictions. & Limited analysis. Draws few connections. Many missed opportunities for deeper thinking. & Lacks any real analysis. No connections made. \\
\midrule
\textbf{Structure \& Organization} & Ideas are organized in a logical, coherent manner. The flow enhances the argument and understanding. & Ideas are mostly well organized. A few minor structural issues but they don't majorly impede understanding. & Organization is somewhat clear, but reader may get lost at times. Some areas lack coherence. & Ideas and sections are disorganized. Hard for the reader to follow. & Lacks any clear structure or organization. Very hard to follow. \\
\midrule
\textbf{Clarity \& Language} & Writing is clear, concise, and free of jargon. No grammatical or stylistic errors. & Writing is mostly clear with minor stylistic or grammatical issues. Might have occasional jargon. & Writing is somewhat clear but can be verbose or have some jargon. Several grammatical issues. & Writing is often unclear, with frequent use of jargon and many grammatical issues. & Writing is consistently unclear. Pervasive jargon and grammatical errors. \\
\bottomrule
\end{tabularx}
\label{tab:synthesis-rubric}
\end{table}

\subsection{Book Reports}

This semester, you will write three book reports, each contributing 15\% to your course grade. The book report is another opportunity to apply and hone the skills you are developing through the AI-assisted peer review, synthesis, and annotated bibliographies for the weekly readings. Here are some guidelines to help you:
    \begin{itemize}
        \item Each report should be limited to twelve pages, using double-space and 12-point Times New Roman (or equivalent) font. Be sure to include page numbers. (I care less about the font than the identical nature of the word counts with these limits.)
        \item Organize your report by chapter. For each chapter, write a brief annotated bibliography, focusing on summarizing the main arguments, theories, and concepts, just as you do in your weekly assignments.
        \item Concisely summarize each chapter; avoid superficially skimming through each chapter.
        \item Focus on summarizing the most essential points for each chapter, by chapter, and avoid disconnected sentences or bullet points.
        \item After summarizing each chapter, synthesize the entire book in the final 1.5–2 pages. This synthesis should highlight the main arguments and theories of the book, discuss its implications for the theory and practice of Public Administration (PA), and explain why this work is essential to the field or your area of the field. This is like the weekly synthesis assignment.
        \item Properly cite each chapter in your book report.
        \item Upload your reports to Canvas by the deadline; late reports will not be accepted.
    \end{itemize}

\subsubsection*{Book Report Discussion Boards}
We will use online Discussion Boards for book report discussions to encourage peer engagement and collaborative learning. Each book has its own Discussion Board (DB) with several prompts for you to address. This is another chance to hone and refine the peer-review skills you learn and use in your weekly reading assignments. Here’s how to participate:

    \begin{itemize}
        \item Upload your book report by the due date and time (11:59 p.m.)
        \item Address at least two prompts in the DB, providing thoughtful responses that encourage discussion.
        \item Respond to at least one other student’s comments (can be on any prompt) using a peer review approach like what you practice in the weekly assignments.
        \item Upload your comments to the DB by Friday of each week (also at 11:59 p.m.)
    \end{itemize}

The discussion board is a virtual platform where you can share your thoughts about the book, engage with your peers, and deepen your understanding of the material. Avoid quoting directly from the book or your report, as it may affect your grade. Instead, use your own words to summarize the main points and provide your own insights. You can also use the DB to ask questions, share resources, and discuss the book’s implications for public administration.

\subsection{Concentration Area Literature Review}

Your final assignment is a literature review on a topic related to your concentration area. This should be approximately ten double-spaced pages with 12-point Times New Roman (or equivalent) font. The instructor must approve your topic by October 5th.

A literature review is a comprehensive survey of existing research related to a specific question or topic. Unlike the conventional literature review, you are not required to propose a new research question for this assignment. Instead, you will review the literature on a preexisting question or topic.

\subsubsection*{The Process}
    \begin{enumerate}
        \item \textbf{Selecting the Topic}: Choose a critical, focused, and related topic for your concentration area. The scope of the topic should be manageable within the two months allocated for this task.
        \item \textbf{Annotated Bibliography}: Write an annotated bibliography of the literature on your topic. The annotated bibliography should be 150 words in length, written in your own words, and include a citation for each source. The annotation should summarize the source's central theme and main points and note its relevance to your topic. This will help you summarize and note the relevance of each work, giving you a solid foundation for the literature review. This is for you and will not be turned in for a grade.
        \item \textbf{Synthesis}: Write a synthesis of the literature on your topic. The synthesis should not be a simple summary of each source but rather a critical analysis that identifies patterns, draws connections, and addresses contradictions among the sources. The synthesis encourages you to think critically and analytically about the literature and its implications for the theory and practice of public administration. This is for you and will not be turned in for a grade.
        \item \textbf{Literature Review Structure}: Organize your literature review into distinct sections and use headings and subheadings as appropriate:
            \begin{itemize}
                \item Introduction: Clearly state your topic and briefly outline your literature review.
                \item Body: Summarize the literature, highlighting the main themes and insights. Identify patterns, draw connections, and address contradictions among the sources. Discuss the implications for public administration. Organize your discussion according to publication chronology, approaches, methods, or theories.
                \item Conclusion: Discuss the findings of the literature review. What do we know, and where is the field on this topic?
                \item Bibliography: List all sources cited in the literature review.
            \end{itemize}
    \end{enumerate} 

\subsubsection*{Additional Parameters}
    \begin{itemize}
        \item The literature review should be approximately ten to twelve double-spaced pages (not including front and back matter) with 12-point Times New Roman (or equivalent) font.
        \item Your literature review should discuss the field’s evolution on the chosen topic, from its origins to its current state, including the most influential works in between.
        \item Discuss whether this is an issue/question/topic with coherent theories, approaches, and models or if it is a disjointed, contentious field. You may find some early literature is discredited by later research.
        \item Remember, this paper is significantly longer than your weekly annotated bibliographies, so it’s a chance to delve deeper into the details of the various approaches, methods, and theories you’ve reviewed.
        \item Your review should be written and free of grammatical/spelling errors.
    \end{itemize}   

\subsubsection*{Executive Summary}
In addition to the literature review, each student will prepare an executive summary to share with students in your concentration. The executive summary should be double-spaced and one to two pages long and should include the following:
    \begin{itemize}
        \item A brief overview of the topic and its relevance to public administration
        \item A summary of the main themes and insights from the literature review
        \item A discussion of the implications for public administration
        \item A bibliography of the sources cited in the executive summary
    \end{itemize}

\section{Grades}

Your work in this class will be graded based on four criteria:
    \begin{enumerate}
        \item Thoroughly complete each assignment, address all questions, and participate in class discussions.
        \item Effective use of class materials (and other literature while researching your literature review topic).
        \item Sophisticated substantive content and discussion rather than superficial.
        \item Writing at the graduate level, including proper mechanics, grammar, syntax, and citation style.
    \end{enumerate}

\subsection*{Grading Scale and Grade Weights}  

The grading scale is shown in Table~\ref{tab:grading-scale}. Grades will be given based on the weights in Table~\ref{tab:grade-weights}.

\begin{table}[h]
\centering
\caption{Grading Scale}
\begin{tabular}{llll}
\toprule
\textbf{Grade} & \textbf{Percentage} & \textbf{Grade} & \textbf{Percentage} \\
\midrule
A+ & 98.0 -- 100 & B- & 80.0 -- 81.9\\
A & 92.0 -- 97.9 & C+ & 78.0 -- 79.9\\
A- & 90.0 -- 91.9 & C & 72.0 -- 77.9\\
B+ & 88.0 -- 89.9 & C- & 70.0 -- 71.9\\
B & 82.0 -- 87.9 & & \\
\bottomrule
\end{tabular}
\label{tab:grading-scale}
\end{table}


\begin{table}[h!]
\centering
\caption{Grade Weights}
\begin{tabular}{ll}
    \toprule
\textbf{Assignment} & \textbf{Percentage} \\
\midrule
Weekly Annotated Bibliography, Synthesis, and Reflection & 40\% \\
Three Book Reports & 30\% \\
Concentration Area Literature Review & 25\% \\
Participation & 5\% \\
\bottomrule
\end{tabular}
\label{tab:grade-weights}
\end{table}

\section{Course Schedule}

\subsection*{Week 1: August 24th}
    \begin{itemize}
        \item \textbf{Topic}: Introduction to Public Administration
        \item \textbf{Orientation to the Course}
    \end{itemize}

\subsection*{Week 2: August 31st}
    \begin{itemize}
        \item \textbf{Topic}: Public Administration Theory
        \item \textbf{Administrivia}:
            \begin{itemize}
                \item Reflections on Peer Review
            \end{itemize}
        \item \textbf{Readings for Weekly Assignment and Discussion}:
            \begin{itemize}
                \item \cite{Weber1946}
                \item \cite{Denhardt2015}, Chapters 1--2
                \item \cite{Rosenbloom2008}
                \item \cite{Wilson1887}
            \end{itemize}
        \item \textbf{Submission}: 
                \begin{itemize}
                    \item Annotated Bibliographies, Synthesis Draft, and Peer Review Activity due by Wednesday at 11:59 p.m. 
                    \item Final Synthesis due by Thursday at 11:59 p.m.
                \end{itemize}
    \end{itemize}

\subsection*{Week 3: September 7th}
    \begin{itemize}
        \item \textbf{Topic}: Public Administration in the U.S. Context
        \item \textbf{Administrivia}:
            \begin{itemize}
                \item Reflections on the weekly reading assignment
                \item Check-in on your literature review topic
                \item Check-in on book report
            \end{itemize}
        \item \textbf{Readings for Weekly Assignment and Discussion}:
            \begin{itemize}
                \item \cite{Allison1990}
                \item \cite{Kaufman1969}
                \item \cite{Kettl2020a}
                \item \cite{Overeem2005}
                \item \cite{Denhardt2015}, Chapters 3--4
            \end{itemize}
        \item \textbf{Submission}: 
            \begin{itemize}
                \item Annotated Bibliographies, Synthesis Draft, and Peer Review Activity due by Wednesday at 11:59 p.m. 
                \item Final Synthesis due by Thursday at 11:59 p.m.
            \end{itemize}
    \end{itemize}

\subsection*{Week 4: September 14th} 
    \begin{itemize}
        \item \textbf{Topic}: Street-Level Bureaucrats
        \item \textbf{Administrivia}: \textbf{NO CLASS}: Book report discussion board
        \item \textbf{Book Report}: Due at 11:59 pm for \cite{Lipsky2010} 
        \item \textbf{Discussion Board}: Discussion on \textit{Canvas} with final conversations due by 11:59 p.m. on Friday, September 15th
    \end{itemize}

\subsection*{Week 5: September 21st}
    \begin{itemize}
        \item \textbf{Topic}: Bureaucracy and Organizational Theory
        \item \textbf{Readings}:
            \begin{itemize}
                \item \cite{Brownlow}
                \item \cite{Downs1967}
                \item \cite{Follett1926}
                \item \cite{Kettl2020}
                \item \cite{Schachter2007}
            \end{itemize}
        \item \textbf{Submission}: 
                \begin{itemize}
                    \item Annotated Bibliographies, Synthesis Draft, and Peer Review Activity due by Wednesday at 11:59 p.m. 
                    \item Final Synthesis due by Thursday at 11:59 p.m.
                \end{itemize}
    \end{itemize}


\subsection*{Week 6: September 28th}
    \begin{itemize}
        \item \textbf{Topic}: Bureaucrats and Representative Bureaucracy
        \item \textbf{Administrivia}:
            \begin{itemize}
                \item Check-in on the literature review topic
            \end{itemize}
        \item \textbf{Readings}
            \begin{itemize}
                \item \cite{Behn2001}, Chapter 1
                \item \cite{Denhardt2015}, Chapter 5   
                \item \cite{Krislov1974}, Chapter 1
                \item \cite{MaynardMoody2012}
                \item \cite{Headly2021}
            \end{itemize}
        \item \textbf{Submission}: 
            \begin{itemize}
                \item Annotated Bibliographies, Synthesis Draft, and Peer Review Activity due by Wednesday at 11:59 p.m. 
                \item Final Synthesis due by Thursday at 11:59 p.m.
            \end{itemize} 
    \end{itemize}


\subsection*{Week 7: October 5th}
    \begin{itemize}
        \item \textbf{Topic}: Public Service Values and Ethics
        \item \textbf{Readings}
            \begin{itemize}
                \item \cite{Adams2009}
                \item \cite{Denhardt2015}, Chapter 7
                \item \cite{Frederickson2005}
                \item \cite{Raile2013}
                \item \cite{Yang2016}
            \end{itemize}
        \item \textbf{Submission}: 
                \begin{itemize}
                    \item Annotated Bibliographies, Synthesis Draft, and Peer Review Activity due by Wednesday at 11:59 p.m. 
                    \item Final Synthesis due by Thursday at 11:59 p.m.
                \end{itemize}
    \end{itemize}

\subsection*{Week 8: October 12th}
    \begin{itemize}
        \item \textbf{Topic}: Networked Governance
        \item \textbf{Administrivia}: \textbf{NO CLASS}: Book report discussion board
        \item \textbf{Book Report}: Due at 11:59 P.M. for \cite{Kamarck2007}
        \item \textbf{Discussion Board}: Discussion on \textit{Canvas} with final conversations due by 11:59 pm on Friday, October 13th
    \end{itemize}

\subsection*{Week 9: October 19th}
    \begin{itemize}
        \item \textbf{Topic}: Behavioral Public Administration: Motivation and Leadership
        \item \textbf{Administrivia}: The literature review topic is due by 11:59 p.m. on October 20th.
        \item \textbf{Readings}:
            \begin{itemize}
                \item Motivation
                    \begin{itemize}
                        \item \cite{Herzberg2003}
                        \item \cite{Christensen2017}
                        \item \cite{Lachance2017}
                    \end{itemize}
                \item Leadership
                    \begin{itemize}
                        \item \cite{Fairholm2004}
                        \item \cite{Magee2014}
                        \item \cite{Paarlberg2010}
                        \item \cite{Denhardt2015}, Chapter 8
                    \end{itemize}
            \end{itemize}
        \item \textbf{Submission}: 
                \begin{itemize}
                    \item Annotated Bibliographies, Synthesis Draft, and Peer Review Activity due by Wednesday at 11:59 p.m. 
                    \item Final Synthesis due by Thursday at 11:59 p.m.
                    \item Half should be on motivation and half on leadership.
                \end{itemize} 
        \end{itemize}

\subsection*{Week 10: October 26th}
    \begin{itemize}
        \item \textbf{Topic}: Behavioral Public Administration: Decision-Making and Nudges
        \item \textbf{Readings}:
            \begin{itemize}
                \item \cite{Caiden1981}
                \item \cite{Lindblom1959}
                \item \cite{Nutt2005}
                \item \cite{Pandey2010}
                \item \cite{thaler2009}, Introduction
            \end{itemize}
        \item \textbf{Submission}: 
            \begin{itemize}
                \item Annotated Bibliographies, Synthesis Draft, and Peer Review Activity due by Wednesday at 11:59 p.m. 
                \item Final Synthesis due by Thursday at 11:59 p.m.
            \end{itemize}
    \end{itemize}


\subsection*{Week 11: November 2nd}
    \begin{itemize}
        \item \textbf{Topic}: Public Policy Implementation
        \item \textbf{Readings}:
            \begin{itemize}
                \item \cite{Bardach1977}
                \item \cite{Head2019}
                \item \cite{Kingdon1995}
                \item \cite{May2007}
                \item \cite{Roman2015}
                \item \cite{Denhardt2015}, Chapter 6
            \end{itemize}
        \item \textbf{Submission}: 
                \begin{itemize}
                    \item Annotated Bibliographies, Synthesis Draft, and Peer Review Activity due by Wednesday at 11:59 p.m. 
                    \item Final Synthesis due by Thursday at 11:59 p.m.
                \end{itemize}
    \end{itemize}


\subsection*{Week 12: November 9th}
    \begin{itemize}
        \item \textbf{Administrivia}: \textbf{ZOOM CLASS}  
            \begin{itemize}
                \item This will be a virtual week with a short class on \href{https://fullerton.zoom.us/j/83783106801}{Zoom}.
                \item \href{https://fullerton.zoom.us/j/83783106801}{\texttt{https://fullerton.zoom.us/j/83783106801}}
            \end{itemize}
        \item \textbf{Topic}: Privatization and Contracting
        \item \textbf{Readings}:
            \begin{itemize}
                \item \cite{Brown2016}
                \item \cite{Cohen2008}
                \item \cite{Hefetz2014}
                \item \cite{Jos2010}
                \item \cite{Lamothe2012}
            \end{itemize}
        \item \textbf{Submission}: 
            \begin{itemize}
                \item Annotated Bibliographies, Synthesis Draft, and Peer Review Activity due by Wednesday at 11:59 p.m. 
                \item Final Synthesis due by Thursday at 11:59 p.m.
            \end{itemize}
    \end{itemize}

\subsection*{Week 13: November 16th}
    \begin{itemize}
        \item \textbf{Topic}: Equity in Public Administration
        \item \textbf{Administrivia}: \textbf{NO CLASS}: Book report discussion board
        \item \textbf{Book Report}: Due at 11:59 p.m. for \cite{Gooden2014}
        \item \textbf{Discussion Board}: Discussion on \textit{Canvas} with final conversations due by 11:59 p.m. on Friday, November 17th
    \end{itemize}

\subsection*{Week 14: November 30th}
    \begin{itemize}
        \item \textbf{Topic}: Measuring Performance
        \item \textbf{Readings}:
            \begin{itemize}
                \item \cite{Behn2003}
                \item \cite{Denhardt2015}, Chapter 9
                \item \cite{Douglas2021}
                \item \cite{Marvel2016}
                \item \cite{NicholsonCrotty2004}
            \end{itemize}
        \item \textbf{Submission}: 
                \begin{itemize}
                    \item Annotated Bibliographies, Synthesis Draft, and Peer Review Activity due by Wednesday at 11:59 p.m. 
                    \item Final Synthesis due by Thursday at 11:59 p.m.
                \end{itemize}
    \end{itemize}

\subsection*{Week 15: December 7th}
    \begin{itemize}
        \item \textbf{Topic}: Public Administration in the 21st Century
        \item \textbf{Readings}:
            \begin{itemize}
                \item Diversity, Equity, Inclusion, Belonging, Acceptance
                    \begin{itemize}
                        \item \cite{McCandless2022}
                        \item \cite{Jiang2022}
                    \end{itemize}
                \item Contemporary and Future Public Administration
                    \begin{itemize}
                        \item \cite{Denhardt2015}, Chapters 10--11
                        \item \cite{Marvel2016}
                        \item \cite{Robles2023}
                    \end{itemize}
            \end{itemize}
        \item \textbf{Submission}: 
            \begin{itemize}
                \item Annotated Bibliographies, Synthesis Draft, and Peer Review Activity due by Wednesday at 11:59 p.m. 
                \item Final Synthesis due by Thursday at 11:59 p.m.
            \end{itemize}
    \end{itemize}
       

\subsection*{Week 16: December 14th}
    \begin{itemize}
        \item \textbf{Topic}: Literature Review Due
        \item \textbf{Administrivia}: Literature Review due by 9:45 p.m. on December 14th
    \end{itemize}


\singlespace
\bibliographystyle{apsr}
\bibliography{521}


\end{document}

